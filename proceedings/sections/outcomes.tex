\section{Network outcomes}
\label{outcomes}
The network is defined around a set of concrete outcomes, guiding the progression of the network and its participants. These intended outcomes are summarised briefly below.

\subsection{Goals of HEP and industry}
\label{goals}
The core outcomes of the network are a set of goals, completed on behalf of the network partners by the network participants.  These goals include experiment commmissioning, HEP measurements and industrial results. These goals are typically defined with respect to a specific ESR position, e.g. the goal of ``calibration of ALICE TPC for heavy-ion physics'' associated to Joachim Hansen.

\subsection{RTA whitepapers}
\label{whitepapers}
Participants will write a series of three whitepapers reviewing the current RTA state-of-the-art, expected to be submitted in late 2023. These whitepapers will place a particular emphasis on the ongoing contributions of the network to their respective topics. \par 

The first whitepaper reviews ML applications to HEP in RTA contexts and their corresponding best practices. Such applications range from data-taking, e.g. particle identification algorithms at ALICE, to offline analysis, e.g. anomaly detection for diject resonance searches at ATLAS. \cite{ALICE-PID, ATLAS-dijet}\par
The use of hybrid architectures by LHC experiments is reviewed in the second whitepaper. Tasks such as selection and reconstruction were significantly accelerated during Run 2 of the LHC (2015-2018). Run 3 deployments of hybrid architectures capitalise on this progress, e.g. in the use of FPGAs in the ALICE Central Trigger System and the use of GPUs in the ATLAS High Level Trigger farm. \cite{ALICE-CTS, CMS-HLT-farm}\par
In the third whitepaper, TDAQ systems of LHC experiments are reviewed, with a focus upon best practices for both TDAQ hardware and software. As such, this whitepaper reviews topics ranging from upgrades to the ATLAS TDAQ system, to the Allen framework of the LHCb experiment enabling software trigger operation at a 30~MHz readout rate. \cite{ATLAS-TDAQ, LHCb-Allen}

\subsection{Training of ESRs}
\label{training}
As a key network objective, participants are encouraged to partake in a broad range of training activities, with network resources dedicated to facilitating this. Such activities include the attendance of external specialist industrial training sessions and academic schools, in addition to internally organised workshops and schools (e.g in Section~\ref{previous-events}).

\subsection{Software and digital assets}
\label{software}
The work of the ESRs will generate digital assets (e.g. software packages, data processing tools), with many being applicable beyond narrow academic/industrial applications. The network is therefore committed to making any such digital assets \textbf{F}indable, \textbf{A}ccessible, \textbf{I}nteroperable, and \textbf{R}eusable (FAIR). \cite{FAIR-principles} To implement these commitments, a project on GitHub has been created to host such assets, \url{https://github.com/SMARTHEP}. Other assets and resources will be made available on the network website, \url{https://smarthep.org}.