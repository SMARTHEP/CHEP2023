\section{Network outcomes}
\label{outcomes}
As an obligation to the European Union, intended network outcomes were laid out as a set of deliverables. These deliverables act to a guide the progression of the network and its participants. Key outcomes of the network are discussed in the following sections.

\subsection{Goals of HEP and industry}
\label{goals}
The core outcomes of the network are a set of goals, which are completed on behalf of the network partners by the network participants. These goals vary, with the majority associated to a specific ESR position, for example, the goal of ``calibration of ALICE TPC for heavy-ion physics'' associated to ESR10, Joachim Hansen.

\subsection{RTA whitepapers}
\label{whitepapers}
As a deliverable of the network, the participants will write a series of whitepapers reviewing the current RTA state-of-the-art. The four whitepaper topics are listed in the following subsections and an overview is given for each. At time of writing, these whitepapers are expected to be submmited for publication by the end of 2023.\par

The first network whitepaper provides a review of current ML applications to HEP in RTA contexts and corresponding best practices. Examples of such applications range from data-taking, e.g. in particle identification at the ALICE experiment, to offline analysis, e.g. anomaly detection for diject resonance searches at the ATLAS experiment. \cite{ALICE-PID, ATLAS-dijet}\par

The use of hybrid architectures by LHC experiments is reviewed in the second SMARTHEP whitepaper. Acceleration of tasks such as selection and reconstruction by hybrid archetictures was developed significantly during Run 2 of the LHC (2015-2018). Run 3 deployments of hybird architectures capitalise on this progress, e.g. in the use of FPGAs in the ALICE Central Trigger System and the use of GPUs in the ATLAS High Level Trigger farm. \cite{ALICE-CTS, CMS-HLT-farm}\par 

In the third whitepaper, the TDAQ systems of LHC experiments are reviewed, with a focus upon best practices for both TDAQ hardware and software. As such, this whitepaper reviews topics ranging from upgrades to the ATLAS TDAQ system, to the Allen framework of the LHCb experiment enabling software trigger operation at a 30~MHz readout rate. \cite{ATLAS-TDAQ, LHCb-Allen}\par

%The final whitepaper focuses on the use of RTA techniques at LHC experiments, in particular in the context of RTA-enabled searches and anomaly detection. For example, in dark photons searches carried out by the CMS and LHCb experiments, which require access to signals previously excluded by prior non-RTA frameworks. \cite{CMS-dark-photon, LHCb-dark-photon} To avoid overlap with the whitepaper discussed in Section~\ref{wp-TDAQ-at-LHC}, this whitepaper does not cover the use of RTA approaches in trigger systems. An exception is made for cases wherein the trigger system forms an inherent part of a process, e.g. RTA-enabled searches where trigger systems provide offline-quality data for direct analysis use.  % WAIT FOR UPDATE ON WHITEPAPER 4

\subsection{Training of ESRs}
\label{training}
Funding is allocated within the network for participants to attend relevant training activities. In the case of ESRs, training and career devlopment is discussed in writing their respective PCDP, providing a clear overview of their intended and requested training activities. Examples of such activities include attendance of specialist industrial training sessions and academic schools. Training is also provided within the network, between participants, for example in the First SMARTHEP School on Collider Physics and Machine Learning discussed in Section~\ref{unige-school}.

\subsection{Software and digital assets}
\label{software}
The work of the ESRs will generate a number of digital assets (e.g. software packages, data processing tools), with many being applicable beyond narrow academic/industrial applications. The network is committed to producing making any produced digital assets \textbf{F}indable, \textbf{A}ccessible, \textbf{I}nteroperable, and \textbf{R}eusable (FAIR). \cite{FAIR-principles} Under these guiding principles our assets and results should be open and accessible to the wider community. To implement these commitments, a project on GitHub has been created to host such assets, \url{https://github.com/SMARTHEP}. Other assets and resources will be made available on the network website, \url{https://smarthep.org}.