\section{Network events}
\label{events}
Network events provide participants with unique opportunities to meet, exchange ideas and develop. Whilst these events typically cater to the ESR audience, some events have been made available to additional interested early career researchers working/studying at SMARTHEP institutes. Previous and upcoming network events are briefly summarised below.

\subsection{Previous events}
\label{previous-events}
To commence ESR particpation in the network, a kick-off meeting was held at the University of Manchester in November 2022 to formally introduce the ESRs to the network and discuss network objectives and organisation. Amongst the activities of the kick-off meeting were a visit to the Jodrell Bank Centre for Astrophysics and a review paper-writing course by Scriptoria to train ESRs ahead of writing the whitepapers.

In January of 2023, the First SMARTHEP School on Collider Physics and Machine Learning was hosted by the University of Geneva. The school provided a varied programme, including lectures on experimental physics at collider experiments by Anna Sfyrla and on theoretical physics \& Monte Carlo event generators by Torbj\"orn Sj\"ostrand. Additionally, hands-on lessons in machine learning were led by Maurizio Pierini, with seminars also given on multimessenger astronomy and the CERN experimental programme by Teresa Montaruli and Jamie Boyd respectively.

\subsection{Upcoming events}
\label{upcoming-events}
Upcoming network events aim to guide the governance of the network, develop ESR expertise and deepen collaborations between participants. The network assembly serves as an annual forum in which to make significant decisions on network policy and governance. Technical hackathons, foreseen for autumn $2023$, give ESRs the opportunity to learn state-of-the-art techniques through their direct application. Hackathons also act as effective team-building activities, promoting problem-solving, rapid prototyping and project management skills, which ESRs can then apply to their own research projects. Accelerator and ML bootcamps, proposed for summer $2024$, will formalise the practical experience of ESRs. Collaboration with industry will culminate in an industry applications school late in the network, by which time it is expected that all ESR industry secondments will have been completed.

