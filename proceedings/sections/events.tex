\section{Network events}
\label{events}
%REPHRASE
Network events form the backbone of the SMARTHEP network, giving participants unique opportunities to meet, exchange ideas and develop. Whilst the events of the network are focused on an ESR audience, SMARTHEP has made many events available to additional interested early career scientists working/studying at SMARTHEP institutes.

\subsection{SMARTHEP Kick-off Meeting}
\label{kick-off}
To commence ESR particpation in the network, a kick-off meeting was held at the University of Manchester from ${21}^\text{st}$ to ${25}^\text{th}$ November 2022, formally introducing the ESRs, and discussing objectives and organisation of the network. The meeting also served as an opportunity for participants from all sides of SMARTHEP to meet in person, network and discuss ideas. To facilitate this, a visit to the Jodrell Bank Centre for Astrophysics was held, and a review paper-writing course was provided by Scriptoria to train ESRs ahead of the writing of the 4 RTA whitepapers.


\subsection{First SMARTHEP School on Collider Physics and Machine Learning}
\label{unige-school}
The First SMARTHEP School on Collider Physics and Machine Learning was hosted by the University of Geneva from ${10}^\text{th}$ to ${13}^\text{th}$ January 2023. The school provided a varied programme, including lectures on experimental physics at collider experiments by Anna Sfyrla and on theoretical physics \& Monte Carlo event generators by Torbj\"orn Sj\"ostrand. Additionally, hands-on lessons in machine learning were led by Maurizio Pierini, with seminars also given on multimessenger astronomy by Teresa Montaruli and the CERN experimental programme by Jamie Boyd.

\subsection{Upcoming events}
\label{upcoming}
At time of writing, the network still has two years ahead of it, in which further network events are planned. These will guide the governance of the network, develop ESR expertise and deepen collaborations with industry.\par
The network assembly sits as a cornerstone of network policy and governance.\par
Accelerator bootcamps in specific aspects of RTA . These are likely to take place in the summer of 2024. An industry applications school will round off the collaboration with industry, by which time it is expected that ESRs will have completed their industry secondments.

