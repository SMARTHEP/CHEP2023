\section{Network events}
\label{events}
Network events form a backbone of the SMARTHEP network, giving participants unique opportunities to meet, exchange ideas and develop. Whilst the events of the network are focused on an ESR audience, SMARTHEP has made many events available to additional interested early career scientists working/studying at SMARTHEP institutes.

\subsection{SMARTHEP Kick-off Meeting}
\label{kick-off}
To mark the commencement of the ESR positions, a kick-off meeting was held at the University of Manchester from ${21}^\text{st}$ to ${25}^\text{th}$ November 2022. This meeting formally introduced ESRs to the network, discussing the objectives and organisation of the network, in addition to introducing the topics-of-work of each ESR. The meeting also served as an opportunity for participants from all sides of SMARTHEP to meet in person, network and discuss ideas. To facilitate this, a visit to the Jodrell Bank Centre for Astrophysics was held. A review paper-writing course was organised by academic writing consultancy group Scriptoria, to train ESRs ahead of the writing of the 4 whitepapers discussed in Section~\ref{whitepapers}.


\subsection{First SMARTHEP School on Collider Physics and Machine Learning}
\label{unige-school}
The First SMARTHEP School on Collider Physics and Machine Learning was hosted by Universit\'e de Gen\`eve between ${10}^\text{st}$ to ${13}^\text{th}$ November 2022. The school delved into many aspects directly relevant to the network, through a varied programme including:
\begin{itemize}
    \item Lectures on experimental physics at collider experiments by Anna Sfyrla.  
    \item Lectures on theoretical physics and Monte Carlo event generators by Torbj\"orn Sj\"ostrand.
    \item Hands-on lessons in machine learning by Maurizio Pierini.
    \item Evening seminars on multimessenger astronomy by Teresa Montaruli and the CERN experimental programme by Jamie Boyd.
\end{itemize}
The school also directly succeeded the mid-term meeting of the network with the EU Project Officer, a key stage in the lifecycle of the network, wherein the progress and trajectory of the network was evaluated by the European Union.


\subsection{Upcoming events}
\label{upcoming}
At time of writing, the network still has two years ahead of it, in which further network events are planned. These will guide the governance of the network, develop ESR expertise and deepen collaborations with industry.\par
The network assembly sits as a cornerstone of network policy and governance.\par
Accelerator bootcamps in specific aspects of RTA . These are likely to take place in the summer of 2024.\par
An industry applications school will round off the collaboration with industry. It is intended that most ESRs will have completed their industry secondments by this time.

