%%%%%%%%%%%%%%%%%%%%%%% file template.tex %%%%%%%%%%%%%%%%%%%%%%%%%
%
% This is a template file for Web of Conferences Journal
%
% Copy it to a new file with a new name and use it as the basis
% for your article
%
%%%%%%%%%%%%%%%%%%%%%%%%%% EDP Science %%%%%%%%%%%%%%%%%%%%%%%%%%%%
%
%%%\documentclass[option]{webofc}
%%% "twocolumn" for typesetting an article in two columns format (default one column)
%
\documentclass{webofc}
\usepackage[varg]{txfonts}   % Web of Conferences font
%
% Put here some packages required or/and some personnal commands
%
\usepackage{multirow}
\usepackage{lineno} % remove in final version
\linenumbers        % remove in final version
%
\begin{document}
%
\title{The SMARTHEP European Training Network}
%
% subtitle is optional
%
%%%\subtitle{Do you have a subtitle?\\ If so, write it here}
\author{\firstname{James Andrew} \lastname{Gooding}\inst{1}\fnsep\thanks{\email{Jamie.Gooding@cern.ch}} \and
        \firstname{Leon} \lastname{Bozianu}\inst{2} \and
        \firstname{Carlos} \lastname{Cocha Toapaxi}\inst{3} \and
        %\firstname{Caterina} \lastname{Doglioni}\inst{2,4} \and
        \firstname{Pratik} \lastname{Jawahar}\inst{4} \and
        \firstname{Micol} \lastname{Olocco}\inst{1}
        } 

\institute{Fakult\"at Physik, Technische Universit\"at Dortmund, Dortmund, Germany \and
           D{\'e}partement de Physique Nucl{\'e}aire et Corpusculaire, Universit{\'e} de Gen{\`e}ve, Geneva, Switzerland \and
           Physikalisches Instit\"ut, Ruprecht-Karls-Universit\"at Heidelberg, Heidelberg, Germany \and
           %Faculty of Physics, Lund University, Lund, Sweden \and
           Department of Physics and Astronomy, University of Manchester, Manchester, United Kingdom
          }

\abstract{%
\textbf{S}ynergies between \textbf{MA}chine learning, \textbf{R}eal-\textbf{T}ime analysis and \textbf{H}ybrid architectures for efficient \textbf{E}vent \textbf{P}rocessing and decision-making (SMARTHEP) is a European Training Network, training a new generation of Early Stage Researchers (ESRs) to advance real-time decision-making, driving data-collection and analysis towards synonymity.

SMARTHEP brings together scientists from major LHC collaborations at the frontiers of real-time analysis (RTA) and key specialists from computer science and industry. By solving concrete problems as a community, SMARTHEP will further the adoption of RTA techniques, enabling future High Energy Physics (HEP) discoveries and generating impact in industry.

ESRs will contribute to European growth, leveraging their hands-on experience in machine learning and accelerators towards commercial deliverables in fields that can profit most from RTA, e.g. transport, manufacturing, and finance.

This contribution presents the training and outreach plan for the network, and is intended as an opportunity for further collaboration and feedback from the CHEP community.
}
%
\maketitle
%

\section{Introduction}
\label{intro}
The \textbf{S}ynergies between \textbf{MA}chine learning, \textbf{R}eal-\textbf{T}ime analysis and \textbf{H}ybrid architectures for efficient \textbf{E}vent \textbf{P}rocessing and decision making (SMARTHEP) European Training Network is a European Union Horizon-funded training network, with a focus on the real-time analysis techniques deployed in high energy physics (HEP) research and in industry. The network centres around 12 doctoral students across Europe employed as Early Stage Researchers (ESRs) within the Marie Sk{\l}odowska-Curie Actions (MSCA) framework. The network commenced in September 2021 and will conclude in September 2025, with each ESR position spanning the 3 years from September 2022 to September 2025.\par
This proceedings sets out the principles, approach and work of the SMARTHEP network, as presented on 9${}^\text{th}$ May 2023 at CHEP2023.

\section{SMARTHEP as a European Training Network}
\label{network}

\begin{table}
    \centering
    \caption{The 18 organisations which form the SMARTHEP network, listed by organisation type.}
    \label{partners}       
    \begin{tabular}{ll}
    \hline
    Category & Partners \\\hline
    Universities        & Lund University, TU Dortmund \\
    Research institutes & CERN, NIKHEF, CNRS \\
    Industry partners   & Ximantis, Verizon Connect \\
    --- & --- \\\hline
    \end{tabular}
\end{table}


\subsection{Partnerships with industry}
\label{sec-2}
Don't forget to give each section, subsection, subsubsection, and
paragraph a unique label (see Sect.~\ref{sec-1}).


\section{Real-time analysis}
\label{rta}
HEP and industry share a common challenge—the rapid processing of large amounts of data. \cite{hu-big-data} Recent advances in computing have enabled the possibility of processing data in real-time, i.e. as data is collected (also commonly referred to as ``online'' processing). \cite{real-time-computing} Real-time analysis (RTA) is thus an umbrella term for such approaches, encompassing a number of state-of-the-art data processing techniques. By processing data online, resources (computing power, storage space, energy, etc.) can be saved and new insights can be obtained from the data recorded, as shown in Figure~\ref{rta-diagram}.\par


\begin{figure*}[h!]
    \centering
    % Use the relevant command for your figure-insertion program
    % to insert the figure file. See example above.
    % If not, use
    %\vspace*{5cm}       % Give the correct figure height in cm
    \includegraphics[width=\linewidth]{/Users/jgooding/Documents/SMARTHEP/CHEP2023/CHEP2023/proceedings/figures/rta-diagram.pdf}
    \caption{Traditional and real-time analysis approaches to data processing. Traditional approaches rely on recording all data and processing this offilne. Real-time analysis applies the reverse approach, processing data as it is produced, recording only the relevant processed information, enabling larger volumes of processed data to be stored.}
    \label{rta-diagram}       % Give a unique label
\end{figure*}
RTA techniques have seen widespread adoption across HEP, in particular amongst trigger and data acquisition (TDAQ) systems. Since it is not possible to record a full detector readout and carry out event reconstruction at the LHC collision rate of {40}{ MHz}, triggers must be devised to select only those events relevant to the physics goals of an experiment. Such triggers conventionally consist of a hardware system making coarse decisions from partial detector readout, and a staged software trigger, applying gradually more finely-grained selections with increasingly more detailed reconstructions of events.\par
In industry, the limitations of the scale of ``big data'' looms over many applications of computing. However, \par
Successful implementation of RTA approaches rests upon the . In particular, in the areas of machine learning and hybrid architectures, which are summarised in Sections~\ref{machine-learning}~\&~\ref{hybrid-architectures}.

\subsection{Machine learning}
\label{machine-learning}
ML, a catch-all term for a family of techniques and technologies wherein algorithms are trained to analyse data, enables rapid decision-making and pattern recognition across a broad range of use cases. \cite{intro-ml} In HEP, the adoption of ML began with classifiers for offline physics analysis, since widening to a variety of classification and pattern recognition/anomaly detection techniques for use online and offline. \cite{albertsson-ml}\par

ML classifiers are algorithms classed with classification, with Boosted Decision Trees (BDTs) and Neural Networks (NNs) amongst those most commonly employed in HEP. The use of BDTs in signal selection for offline physics analysis has become standard practice, e.g. suppresion of combinatorial background in the LHCb experiment measurement of the $B_s^0-\bar{B}_s^0$ oscillation frequency. \cite{delta-ms} Classifiers can also aid tasks such as event reconstruction, e.g. the CMS boosted event shape tagger, a NN trained to discriminate between possible $t$, $W^\pm$, $Z^0$ and $H$ candidates within an event. \cite{CMS-best} In industry, ML classifiers are used frequently in content deliver\par

ML techniques can also be applied to pattern recognition and anomaly detection—tasks which cannot be realistically carried out on the scales of data presently being analysed. In industry, this is applied intuitively to fraud detection, wherein subtle details in data such as the transactions of a customer. In HEP \cite{anomaly-hep}

\subsection{Hybrid architectures}
\label{hybrid-architectures}
Typical computational resources consist of Central Processing Units (CPUs), processors which are designed for general purpose computation (i.e. varied tasks with significant variety in computing/memory resources), with large on-board memory and often multiple processing cores. Alternative architectures, such as those shown in Figure~\ref{architectures}, can be applied in conjunction or in place of CPU architectures to accelerate processing where tasks are not well-suited are to CPU architectures alones. \par

\begin{figure*}[h!]
    \centering
    % Use the relevant command for your figure-insertion program
    % to insert the figure file. See example above.
    % If not, use
    %\vspace*{5cm}       % Give the correct figure height in cm
    \includegraphics[width=0.9\linewidth,clip]{/Users/jgooding/Documents/SMARTHEP/CHEP2023/CHEP2023/proceedings/figures/architectures-diagram.pdf}
    \caption{Comparison of CPU, GPU and FPGA architectures. A CPU is typically formed of several cores, each containing computational, control and cache resources, and a centralized cache, memory and input/output (IO) interface. GPUs are formed similarly, with centralized cache, memory and IO, and many multiprocessors; however, each multiprocessor contains a greater proportion of computational resources, which are themselves partitioned to perform tasks in parallel. FPGAs are structured in a radically different way, with memory and IO connected to many interlinked control blocks, each primarily formed of simpler logic gate arrangements and often accompanied by a small cache.}
    \label{architectures}       % Give a unique label
\end{figure*}

Field-programmable gate arrays (FPGAs) are integrated circuits which can be programmed by the user for specific tasks. \cite{fpgas-intro} An FPGA consists of many simple logic circuits with a small amount of on-board memory, formed into logic blocks, each of which are connected via switch matrices. \cite{fpgas-book} FPGAs are thus well-suited to bespoke, computationally light but highly parallelisable tasks, such as low-level trigger decisions, wherein simple selections must be made rapidly. \cite{duarte-fpgas} In industry, FPGAs have seen adoption . However; a lack of on-board memory limits the deployment of FPGAs to more intensive tasks.\par
Graphical Processing Units (GPUs) can also be employed to . \cite{vomBruch-gpus} Historically, the use of GPUs has been 


\section{Early stage researchers}
\label{esrs}
The 12 ESRs form the core of the network, with the training and partnerships providing a scaffold for the completion of their respective outcomes. These ESRs work with HEP and industry partners throughout the 3 years of their doctoral study.\par
e\par
Shortly after the commencement of the ESR positions, each ESR supervisor completed a Personal Career Development Plan with their ESR. This document details the 

\subsection{Structure of ESR positions}
\label{esr-structure}
Each ESR is enrolled as a doctoral student at a partner university for 3 years. This university is typically also the primary institution of the ESR, though the industry-centred ESR positions 2 and 10 (discussed in Section~\ref{esr-positions}) are carried out with industry partners near to the university of enrollment. Where partner universities require a longer period of study or typically provide extensions to a 3 year study programme, commitments to further enrollment are provided by each partner university where necessary. At the end of the period of doctoral study, each ESR will receive a doctorate in particle physics from their university of enrollment.\par
ESRs\par
a
\begin{figure*}[h!]
    \centering
    % Use the relevant command for your figure-insertion program
    % to insert the figure file. See example above.
    % If not, use
    %\vspace*{5cm}       % Give the correct figure height in cm
    \includegraphics[width=\linewidth,clip]{/Users/jgooding/Documents/SMARTHEP/CHEP2023/CHEP2023/proceedings/figures/esr-diagram.pdf}
    \caption{Diagram of the structure of a SMARTHEP ESR position. Each ESR is enrolled (i.). During the course of their enrolment, they will undertake secondments with network partners in HEP (ii.) and industry (iii.). Through the combination of their primary and secondment work, each ESR will achieve several goals in HEP (iv.) and industry (v.), as discussed in further detail in Section~\ref{goals}.}
    \label{esr-diagram}       % Give a unique label
\end{figure*}

Each ESR will undertake a secondment in HEP during their studies, either at another partner university or a partner research institute. This is generally organised to last for 6 months, though this varies from project to project.\par

In addition to their time working in HEP, each ESR will also spend time working in industry, typically as a secondment of 3-4 months (though this is again flexible to the specific programme of each ESR position).

\subsection{Examples of ESR positions}
\label{esr-examples}
The 12 ESR positions of the network are summarised in Table~\ref{esr-positions}.\par
\begin{table}[h!]
    \footnotesize
    \centering
    \caption{Please write your table caption here}
    \label{esr-positions}       
    \begin{tabular}{p{3.1cm}p{3.25cm}p{2.85cm}p{2cm}}
    \hline
    \multirow{2}{*}{\centering Early-stage researcher} & \multirow{2}{*}{\centering Primary affiliation} &  \multicolumn{2}{c}{Secondments} \\
    & & \multicolumn{1}{c}{HEP} & \multicolumn{1}{c}{Industry} \\\hline 
    ESR1: Patin Inkaew & University of Helsinki & CERN & Verizon Connect \\
    ESR2: Laura Boggia & IBM France (enrolled at University Sorbonne) & CNRS & IBM France \\
    ESR3: Leon Bozianu & University of Geneva & IGFAE
     & Lightbox \\
    ESR4: Sofia Cella & University of Geneva & CNRS & Lightbox \\
    ESR5: Fotis Giasemis & Sorbonne University & CERN & Ximantis \\
    ESR6: Daniel Magdalinski & VU Amsterdam/NIKHEF & TU Dortmund & Point 8 \\
    ESR7: Jamie Gooding & TU Dortmund & CERN & Ximantis \\
    ESR8: Micol Olocco & TU Dortmund & CERN & Point 8 \\
    ESR9: Carlos Cocha & University of Heidelberg & TU Dortmund/IGFAE & Verizon Connect \\
    ESR10: Joachim Hansen & Lund University & CERN & Ximantis \\
    ESR11: Henrique Pi\~neiro Monteagudo & Verizon Connect (enrolled at University of Bologna) & Sorbonne University/ University of Manchester & Verizon Connect \\
    ESR12: Pratik Jawahar & University of Manchester & CERN & University of Manchester \\\hline
    \end{tabular}
\end{table}
To paint a clearer picture of how the positions take shape, a few examples are taken from Table~\ref{esr-positions}. Jamie Gooding (ESR7) is based at Technische Universität Dortmund, working on real-time event selection at the LHCb Experiment, with a HEP secondment of $\approx 6$ months at CERN to support this work and a 3-4 month industry secondment with Ximantis, working on real-time traffic monitoring. Laura Boggia (ESR2)  \par

\section{Network outcomes}
\label{outcomes}
The network is anchored by a set of concrete outcomes, guiding the progression of the network and its participants. These intended outcomes are summarised briefly below.

\subsection{Goals of HEP and industry}
\label{goals}
The core outcomes of the network are a set of goals, completed on behalf of the network partners by the network participants.  These goals include experiment commmissioning, HEP measurements and industrial results. These goals are typically defined with respect to a specific ESR position, e.g. the goal of ``calibration of ALICE TPC for heavy-ion physics'' associated to Joachim Hansen.

\subsection{RTA whitepapers}
\label{whitepapers}
Participants will write a series of three whitepapers reviewing the current RTA state-of-the-art, expected to be submmited in late 2023. These whitepapers will place a particular emphasis on the ongoing contributions of the network to their respective topics. \par

The first whitepaper reviews ML applications to HEP in RTA contexts and their corresponding best practices. Such applications range from data-taking, e.g. particle identification at ALICE, to offline analysis, e.g. anomaly detection for diject resonance searches at ATLAS. \cite{ALICE-PID, ATLAS-dijet} The use of hybrid architectures by LHC experiments is reviewed in the second whitepaper. Tasks such as selection and reconstruction were significantly accelerated during Run 2 of the LHC (2015-2018). Run 3 deployments of hybrid architectures capitalise on this progress, e.g. in the use of FPGAs in the ALICE Central Trigger System and the use of GPUs in the ATLAS High Level Trigger farm. \cite{ALICE-CTS, CMS-HLT-farm} In the third whitepaper, TDAQ systems of LHC experiments are reviewed, with a focus upon best practices for both TDAQ hardware and software. As such, this whitepaper reviews topics ranging from upgrades to the ATLAS TDAQ system, to the Allen framework of the LHCb experiment enabling software trigger operation at a 30~MHz readout rate. \cite{ATLAS-TDAQ, LHCb-Allen}

\subsection{Training of ESRs}
\label{training}
Funding is allocated within the network for participants to attend relevant training activities. In the case of ESRs, training and career devlopment is discussed in writing their respective PCDP, providing a clear overview of their intended and requested training activities. Examples of such activities include attendance of specialist industrial training sessions and academic schools. Training is also provided within the network, between participants, for example in the First SMARTHEP School on Collider Physics and Machine Learning discussed in Section~\ref{unige-school}.

\subsection{Software and digital assets}
\label{software}
The work of the ESRs will generate digital assets (e.g. software packages, data processing tools), with many being applicable beyond narrow academic/industrial applications. The network is therefore committed to making any such digital assets \textbf{F}indable, \textbf{A}ccessible, \textbf{I}nteroperable, and \textbf{R}eusable (FAIR). \cite{FAIR-principles} To implement these commitments, a project on GitHub has been created to host such assets, \url{https://github.com/SMARTHEP}. Other assets and resources will be made available on the network website, \url{https://smarthep.org}.

\section{Network events}
\label{events}
%REPHRASE
Network events form the backbone of the SMARTHEP network, giving participants unique opportunities to meet, exchange ideas and develop. Whilst the events of the network are focused on an ESR audience, SMARTHEP has made many events available to additional interested early career scientists working/studying at SMARTHEP institutes.

\subsection{SMARTHEP Kick-off Meeting}
\label{kick-off}
To commence ESR particpation in the network, a kick-off meeting was held at the University of Manchester from ${21}^\text{st}$ to ${25}^\text{th}$ November 2022, formally introducing the ESRs, and discussing objectives and organisation of the network. The meeting also served as an opportunity for participants from all sides of SMARTHEP to meet in person, network and discuss ideas. To facilitate this, a visit to the Jodrell Bank Centre for Astrophysics was held, and a review paper-writing course was provided by Scriptoria to train ESRs ahead of the writing of the 4 RTA whitepapers.


\subsection{First SMARTHEP School on Collider Physics and Machine Learning}
\label{unige-school}
The First SMARTHEP School on Collider Physics and Machine Learning was hosted by the University of Geneva from ${10}^\text{th}$ to ${13}^\text{th}$ January 2023. The school provided a varied programme, including lectures on experimental physics at collider experiments by Anna Sfyrla and on theoretical physics \& Monte Carlo event generators by Torbj\"orn Sj\"ostrand. Additionally, hands-on lessons in machine learning were led by Maurizio Pierini, with seminars also given on multimessenger astronomy by Teresa Montaruli and the CERN experimental programme by Jamie Boyd.

\subsection{Upcoming events}
\label{upcoming}
At time of writing, the network still has two years ahead of it, in which further network events are planned. These will guide the governance of the network, develop ESR expertise and deepen collaborations with industry.\par
The network assembly sits as a cornerstone of network policy and governance.\par
Accelerator bootcamps in specific aspects of RTA . These are likely to take place in the summer of 2024. An industry applications school will round off the collaboration with industry, by which time it is expected that ESRs will have completed their industry secondments.



\section{Conclusion}
\label{conclusion}
The network will provide valuable contributions in HEP, particularly to the commisioning and operation of LHC experiments throughout Run 3 of the LHC. Additionally, the network serves to further the adoption of RTA approaches in industry and do demonstrate the value of collaboration between HEP and industry.

\section*{Acknowledgements}
We acknowledge funding from the European Union Horizon 2020 research and innovation programme, call H2020-MSCA- ITN-2020, under Grant Agreement n. 956086

\bibliography{references.bib}


\end{document}

% end of file template.tex

<div id='footer'><table width='100%'><tr><td class='right'><a href='http://fusioninventory.org/'><span class='copyright'>FusionInventory 9.1+1.0 | copyleft <img src='/glpi/plugins/fusioninventory/pics/copyleft.png'/>  2010-2016 by FusionInventory Team</span></a></td></tr></table></div>